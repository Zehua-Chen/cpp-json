\documentclass[11pt, letterpaper]{article}
\usepackage[margin=1in]{geometry}
\usepackage{booktabs}
\usepackage{longtable}
\usepackage{tabu}
\usepackage[scaled=0.92]{helvet}

\setlength{\parindent}{0pt}
\renewcommand{\familydefault}{\sfdefault}

\title{Json Composer Design}
\author{Zehua Chen}
\date{\today}

\newcommand{\objectstart}{o\_s}
\newcommand{\arraystart}{a\_s}
\newcommand{\objecthaskeystring}{o\_h\_ks}
\newcommand{\objecthaskey}{o\_h\_k}
\newcommand{\objecthasvalue}{o\_h\_\_v}
\newcommand{\arrayhasvalue}{a\_h\_v}

\begin{document}
  \maketitle
  
  \section{States}
  
    \begin{description}
      \item[start] initial state of the composer;
      \item[error] the state the composer will be in if an error has occured; 
      \item[finish] the state the compoer will be in after the root object has
      been parsed;  
      \item[object start] the initial start of an object;
      \item[object has key string] the state in which the composer will be in 
      after receiving the string that represents the key;
      \item[object has key] the state in which the composer will be in after 
      receiving the key string and the key value separator 
      token;
      \item[object has value] the state in which the composer will be in after
      receiving a value;
      \item[object ready] the state in which the composer will be in after 
      receiving a value separator and is ready to start 
      taking a new key-value pair; 
      \item[array start] the inital state of an array;
      \item[array has value] the state in which the composer will be in after 
      receiving a value;  
      \item[array ready] the state in which the composer will be in after 
      receiving value separator and ready to take the next element; 
    \end{description}
    
    \begin{tabu} to \linewidth{ X[l] | X[l] }
      \toprule[1pt]
      \textbf{State} & \textbf{Shorthand} \\ \midrule[1pt] 
      start & start \\ \hline
      error & error \\ \hline
      finish & finish \\ \hline
      object start & \objectstart \\ \hline
      object has key string & \objecthaskeystring \\ \hline
      object has key & \objecthaskey \\ \hline
      object has value & \objecthasvalue \\ \hline
      object ready & o\_r \\ \hline
      array start & \arraystart \\ \hline
      array has value & \arrayhasvalue \\ \hline
      array ready & a\_r \\ \bottomrule[1pt]
    \end{tabu}
  
  \section{Inputs}
  
    \begin{tabu} to \linewidth{ X[1,l] | X[3,l] }
      \toprule[1pt]
      \textbf{Input} & \textbf{Description} \\ \midrule[1pt] 
      begin\_obj & true when the token is begin object \\ \hline
      begin\_array & true when the token is begin array \\ \hline
      end\_obj & true when the token is end object \\ \hline
      end\_array & true when the token is end array \\ \hline
      string & true when the token is string \\ \hline
      boolean & true when the token is boolean \\ \hline
      number & true when the token is number \\ \hline
      null & true when the token is null \\ \hline
      non\_string & true when the token is a primitive other than string \\ \hline
      primitive & true when the token is string, boolean, number, or null \\ \hline
      void\_context & true when there is no parent object \\ \hline
      obj\_context & true when the parent is an object \\ \hline
      array\_context & true when the parent is an array \\ \bottomrule[1pt] 
    \end{tabu}
    
  \section{Transitions}
    
    \begin{longtabu} to \linewidth{  X[1,l] | X[1,l] | X[3,l] | X[3,l] }
      \toprule[1pt]
      \textbf{From} & \textbf{To} & \textbf{Condition} & \textbf{Action} \\ \midrule[1pt]
      % start state's outbonding transitions
      \multicolumn{4}{c}{From \textbf{Start} State} \\ \midrule[1pt]
      start & finish & primitive & push stack \\ \hline 
      start & \objectstart & begin\_obj & push stack \\ \hline
      start & \arraystart & begin\_array & push stack \\ \midrule[1pt]
      % object start state's outbonding transitions
      \multicolumn{4}{c}{From \textbf{Object Start} State} \\ \midrule[1pt]
      \objectstart & \objecthaskeystring & string & record key \\ \hline
      \objectstart & \objecthasvalue & end\_obj \& obj\_context & pop stack \\ \hline
      \objectstart & \arrayhasvalue & end\_obj \& array\_context & pop stack \\ \hline
      \objectstart & finish & end\_obj \& void\_context & \\ \hline
      \objectstart & error & else & \\ \hline
    \end{longtabu}
\end{document}